\documentclass[12pt]{article}
\usepackage{mathptmx}
\usepackage{setspace}
\usepackage[margin=1.0in]{geometry}
\doublespacing

\title{Analysis of Joker}
\author{Michael Cardiff}
\date{August 10, 2020}

\begin{document}
\maketitle
The movie \textit{Joker}, directed by Todd Phillips is an interesting movie, to the public it is presented as a "superhero" movie, but ends up being more about a more mentally disturbed man. The movie follows the life of the man who eventually becomes the titular Joker, but his actual name is Arthur Fleck. Arthur lives with his mother Penny, who will become more important as we investigate Arthur's psychopathology further. It is important to note that Penny is not Arthur's biological mother, so any abnormal behavior do not stem from genetics. Another thing that we learn about Arthur's past is that he was abused by an ex partner of his mother, which led to Penny being institutionalized. The relevant plot to the movie is that Arthur goes through a series of misfortunes in his life that end up with him finding out he is adopted. Upon finding this out, Arthur ends up killing his mother. At the same time, he ends up being invited on a talk show hosted by Murray Franklin, but he is only invited to be made fun of by Murray. Arthur also ends up killing Murray. We will now explore the specific abnormal behaviors which Arthur exhibits.
\par
One interesting behavior seen throughout the movie is Arthur's laughing. In the movie this is explained by an unnamed mental disorder for which he has cards which he hands out to people who ask why he is laughing. Arthur also experiences hallucinations after he is informed that he will no longer be provided medicine by the government. He hallucinates that he is in a relationship with a woman who lives in the same building as him and his mother. He hallucinates that they go on dates together, and even that she attended his stand-up performance, which in reality, goes horribly. Another hallucination he has is earlier in the movie, where he sees himself going on Murray's show, where he is praised for seemingly no reason, this does not match what really happens when he eventually goes on the show. Arthur also shows some abnormal coping methods throughout the movie. After getting fired from his job as a clown, Arthur ends up killing three men. After leaving the train, he goes to a gas station and seems to cope with the trauma by dancing. He does the same thing later in the movie right before he goes on to Murray's show as the Joker. During his time on Murray's show he rants about how society does not care about him, and that people step all over him because he is not noticed because of his mental illness, leading him to kill Murray as well. Arthur is arrested on sight, but because of nearby riots, he is broken out and of course dances to cope with it. I will now use these behaviors to diagnose Arthur.
\par
First and foremost I believe that Arthur is a victim of factitious disorder by proxy, with the perpetrator being his mother Penny. Throughout the movie, she tells Arthur that he is special because of this disorder, even though it does not exist. However, this does not rule out other mental disorders, as there should be a reason that Arthur finds things like murder funny. As for Arthur, he shows many symptoms of Schizophrenia in terms of positive and the dissociative symptoms, but not so much the negative symptoms. Arthur experiences distortion of normal behavior as seen when he is preparing for his stand-up set. He is seen laughing when literally no one else is laughing. His hallucinations of being on Murray's show initially show he has delusions of everyone agreeing with him. His more disorganized symptoms show when he is on Murray's show. After he comes on, he is almost immediately made fun of, even though he is dressed as a clown. He goes on to rant about how society is against him, and that any one of the people who laughed at him would step all over him, even though those people do not know him at all and have likely never met him. One negative symptom that Arthur may demonstrate is avolition, as he shows very little remorse for killing the three Wayne-Corp men, and obviously none for killing his mother either. While these definitely put Arthur on the Schizophrenic spectrum, we need to specify where exactly he is. If the movie alone was the only information we had, it would be difficult, as the movie takes place over the span of no more than a week. But since, as viewers from the outside world where Joker is a known character, we know his symptoms last for much longer than the run time of the movie, so in this case I would say he has Schizophrenia. If only the movie is to be taken into account, then Arthur would most likely either have Schizophreniform disorder or full blown Schizophrenia. Now we will move onto etiological theories which may explain Arthur's diagnosis.
\par
There are a number of stressors present in the movie which drive Arthur to the point he is at at the end of the movie. This makes me think that a psychological stressor model may be key in the etiology of Arthur's disorder. Genetic influences may be important but we know nothing of Arthur's biological parents. Seeing as Arthur is not real, we cannot determine if he has structural brain damage. We can speculate when it comes to the medicine that Arthur was taking as to whether it affects the neurotransmitter balance/imbalance. Since the hallucinations only occur after Arthur has stopped taking his medicine, I would believe that the medicine he was taking was a dopamine antagonist based on the dopamine hypothesis. Due to the presence of Arthur finding out he is adopted, killing five people, and losing his job, I would have to say that psychological stressors played a major role in the development of his psychosis in such a short period of time. However, one can argue that many of these would not have happened if Arthur had been able to continue his medication and continue the very limited therapy he was in, except for maybe him losing his job. Even though some of it is very much speculation, the dopamine hypothesis and psychological stressor models provide an explanation for the etiology of Arthur's schizophrenic spectrum disorder, whether it be schizophrenia or schizophreniform disorder.
\par
To conclude, I will summarize Arthur's behaviors and his diagnosis. Arthur demonstrated a pattern of apathetic emotions, warped views of the world, and hallucinations that persisted over a period of a week through the movie. Given outside information about the Joker, we know that this pattern persists, so he would be diagnosed with schizophrenia. However, if we take only the knowledge presented in the movie, Arthur would be diagnosed with either schizophreniform disorder or schizophrenia. 
\end{document}
