\documentclass[12pt]{article}

\title{PSYC 303 Notes}
\author{Michael Cardiff}
\date{Summer 2020}

\begin{document}
\maketitle
\section{Week 1 Notes}
\paragraph{Video 2}
Two general models that explain certain psychological behavior:
\begin{enumerate}
\item One Dimensional
\item Multi Dimensional
\end{enumerate}
1-D Models usually attribute an abnormality to a single cause, ie
schizophernia is caused by a chemical imbalance. Multidimensional
models tend to use many of these in concert with each other, some
models use the following influences:
\begin{enumerate}
\item Biological
\item Behavioral
\item Emotional
\item Cognitive
\item Social and Cultural
\item Developmental
\end{enumerate}
\subsection{Biological}
Major implicaation is genetics. Chromosomes are what contain the
genes. 46 chromosomes in 23 pairs. Half from mom half from dad. The
last pair determine sex. They determine physical and mental attributes
of a person.
\subsubsection{Genes}
There are dominant and recessive genes. If you have one of a dominant
gene, you WILL have the disease. Recessive genes will need 2, you are
called a carrier. Most psychological disorders are polygenetic, no
single gene determines it. Genes contribute less than 50\% to
psychopathology. Another factor is environment. Genes interact with
environment to produce these psychological effects. Two models:
\begin{enumerate}
\item Diathesis stress model
  \begin{enumerate}
  \item Diathesis: inherited tendedncy or vulnerability
  \item Stress: Life event(s)
  \item People with the same stressor do not produce the same
    diseases, this is because of the different amount of diathesis
    they experience.
  \item This model is useful for substance abuse disorders
  \item Depression is an example of this genetic vulnerability
    intertwining with life examples, check study on slides at 12:00
  \item Neither genes alone or stress alone determine
  \end{enumerate}
\item Reciprocal Gene Environment Mode
  \begin{enumerate}
  \item Two way interaction between genes and stress
  \item genes can make people more likely to seek out different life
    experiences.
  \item People who have depression might be more likely to seek out
    depression triggering relationships
  \item Enovironmental influences can override the genes
  \item Rat studies with depressed moms are better with calm moms,
    non-genomic inheritance. Determined in first week for rats.
  \item Demonstrates critical periods of development. Also seen in
    human babies adopted to dysfunctional families. 
  \item Essentially, genes are not the whole story. 
  \end{enumerate}
\end{enumerate}
\subsubsection{Neuroscience}
Neuroscience discusses the role of the nervous system and
behavior. There are two branches of the human nervous system:
\begin{enumerate}
\item The central nervous system, comprising of the brain and spinal cord
\item The peripheral nervous system, somatic and autonomic branches
\end{enumerate}

Here is a overview of neuroscience and brain structure:
\begin{enumerate}
\item Main Structures:
  \begin{enumerate}
  \item Brain stem:
    \begin{enumerate}
    \item Basic functions like attention
    \item Contains hindbrain midbrain, thalamus, hypothalamus
    \end{enumerate}
  \item Forebrain
    \begin{enumerate}
    \item Controls more complex functions like memory
    \item Limbic system basal ganglia
    \end{enumerate}
  \end{enumerate}
\item Hindbrain:
  \begin{enumerate}
  \item Medulla - Heart rate, blood pressure
  \item Pons - Coordinates movement and regulates sleep and arousal
  \item Cerebellum - involved in physical coordination in short intervals
  \end{enumerate}
\item Midbrain:
  \begin{enumerate}
  \item Coordinates movement with sensory input
  \item Contains part of the reticular activating system 
  \end{enumerate}
\item Basil Ganglia
  \begin{enumerate}
  \item Motor Behavior based on cerebellum
  \end{enumerate}
\item Limbic System
  \begin{enumerate}
  \item Regulate emotional experiences
  \item Amygdala - regulates perception and reaction to aggression and
    fear, and involved in emotional learning
  \item Thalamus - receives and integrates sensory information, relays
    to cerebral cortex.
  \item Hypothalamus - Eating, drinking aggression, sexual
    activity. Releases hormones
  \item Hippocampus - Long term declarative memory and special
    processing. involves facts
  \end{enumerate}
\item Forebrain
  \begin{enumerate}
  \item Sensroy emotional and cognitive processing
  \item Cerebral cortex contains two hemispheres:
    \begin{enumerate}
    \item Left: verbal and other cognitive processes
    \item Right: Perceives world and creates images
      Each has 4 lobes:
      \begin{enumerate}
      \item Frontal - Thinking and reasoning
      \item Parietal - Touch recogntion
      \item Occiptal - Integrates visual input
      \item Temporal - Temporal - Sights and sounds, storage for long
        term memory
      \end{enumerate}
    \end{enumerate}
  \end{enumerate}
\item Somatic NS:
  \begin{enumerate}
  \item Controls volunary movement
  \end{enumerate}
\item Autonomic NS:
  \begin{enumerate}
  \item Sympathetic and parasympathetic branches
  \item Regulates cardiovascular system and body temp
  \item Regulates endocrine system.
  \end{enumerate}
\item Sympathetic division deploys body in stressful situations,
  effects include increased heart rate and increased respiration,
  mobilize body in a situation. Useful when there is immediate
  danger. 
\item If these effects become too much, the parasympathetic system
  takes over. 
\item People with anxiety use outside methods to activate the
  parasympathetic branch.
\item Endocrine system:
  \begin{enumerate}
  \item Regulates release of hormones
  \item hormones active in very small amounts
  \item The endocrine system and nervous system are integrated in the
    HPA axis
  \end{enumerate}
\item Neurons:
  \begin{enumerate}
  \item Soma - cell body
  \item Dendrites - Branches that receive messages from other neurons
  \item Axon - Trunk that sends messages to other neurons
  \item Myelin Sheath - Covers the axon that speeds impulses
  \item Terminal Buttons - Forms junction with dendrite of other
    neurons. 
  \item They operate electronically, but communicate chemically
  \item A breakdown of this is on 40:00
  \end{enumerate}
\item Function of main types of neurotransmitters:
  \begin{enumerate}
  \item Agonists - increases effect of a neurotransmitter
  \item Antagonists - inhibits neurons either directly or indirectly
  \item Inverse Agonists - produces opposite effects to a neurotransmitter
  \item Most drugs are agonistic or antagonistic
  \end{enumerate}
\item Types of these:
  \begin{enumerate}
  \item Serotonin - mood, eating, sleeping, arousal
  \item Gamma aminobutyric acid (GABA) - inhibits other neurons,
    involuntary motor action, alcohol inhibits GABA
  \item Glutamate - excititory, opposite of GABA, increases other neurons
  \item Norepinephrine - Adrenaline, autonomic NS, sympathetic
    arousal, fight or flight
  \item Dopamine - Processing motor movements, attention, reinforces
    effects of drugs which are abused. Parkinsins happens with no
    dopamine. Sex and Gaming give dopamine
  \end{enumerate}
\item Manipulation of serotonin is a treatment for depression. Prozac
  is an SSRI. It makes more serotonin available.
\item We have neuroimaging but people had to autopsy people before this.
\item Paul Broca identified the speech area which was damaged in one
  man that spoke bad but not another man.
\item Two types of Neuroimaging:
  \begin{enumerate}
  \item Structural Imaging, actual brain itself. Does the brain look
    the same as someone who does not have this disorder?
  \item Functional Imaging, is the brain doing what it should?
  \end{enumerate}
\item Examining Structre:
  \begin{enumerate}
  \item CAT Scan
    \begin{enumerate}
    \item Computerized axial tomography
    \item Multiple x-rays
    \item Risks
    \end{enumerate}
  \item MRI
    \begin{enumerate}
    \item Magnetic Resonance Imaging
    \item More accurate than a CAT Scan
    \item Radio signals excite the brain tissue
    \item Fewer risks
    \end{enumerate}
  \end{enumerate}
\item Examining function:
  \begin{enumerate}  
  \item PET Scan
    \begin{enumerate}
    \item Positron Emission Tomography
    \item Blood flow changes via a tracer
    \end{enumerate}
  \item SPECT
    \begin{enumerate}
    \item Single photon emission tomography
    \item Different tracer
    \item Less precise than a PET but cheaper
    \item Faster than PET
    \end{enumerate}
  \item These both can help us determine whats happening in the brain
    that causes psych disorders
  \end{enumerate}
\item Implications of Neuroscience for Psychopathology:
  \begin{enumerate}
  \item Relations between brain and abnormal behavior such as OCD and Schizophrenia
  \item Psychosocial influences can change brain structure and function
  \item Therapy can also change brain structure and function along
    with medication
  \end{enumerate}
\end{enumerate}
\subsubsection{Behavioral}
\begin{enumerate}
\item Classical Conditioning (Pavlov)
  \begin{enumerate}
  \item Associatiion of two things that go together
  \item Condition stimulus: Metronome
  \item Unconditioned stimulus: Meat
  \item Conditioned response: Salivation
  \item Exclusivity is important
  \end{enumerate}
\item Operant Conditioning (Skinner)
  \begin{enumerate}
  \item Method of learning to repeat behaviour with desirable
    consequences and suppress behavior with bad consequences.
  \item Reinforcement and Punishment
  \end{enumerate}
\item Learned Helplessness (Seligman)
  \begin{enumerate}
  \item Uncontrollable shock
  \item Anxiety and depression results from exposure to uncontrollable
    punishment which leads to deneralized feelings of helplessness
  \item Lack of control over environment. 
  \item Generalized sense of helplessness, leads to anxiety
  \end{enumerate}
\item Social Learning (Bandura)
  \begin{enumerate}
  \item Modeling and observational learning
  \item Learn behaviors through observation and imitation of others
  \end{enumerate}
\item Prepared Learning
  \begin{enumerate}
  \item Prepearedness inherited from ancestors (biologically driven)
  \item Good evolutionary reasons to learn some associations
  \item People fear social rejection due to a need to hunt in packs.
  \end{enumerate}
\end{enumerate}
\subsubsection{Cognitive Models}
\begin{enumerate}
\item Psychopathology results from maladaptive thinking patterns
\item negative attributional style, hopelessness
\item Dissociation between behavior and consciousness
\item implicit memory - being affected by an event that you do not
  remember, people may not remember car accident but still affected by
  it mentally
\item Stroop Paradigm
  \begin{enumerate}
  \item Name color of word which is printed
  \item People with disorders have trouble identifying colors of
    things in the anxiety block
  \end{enumerate}
\item Emotions
  \begin{enumerate}
  \item To elicit or evoke action
  \item Tied to several forms of psychopathology such as mood disorders
  \item Moods are presistent periods of emotions
  \item Components of emotion - Behavior, Physiology, Cognition
  \item Suppressiing negative emotions increases SNS activity
  \item Dysregulated emotions are key features of many mental disorders
  \end{enumerate}
\item Chronic hostility increases risk for heart disease, efficiency
  of heart pumping is decreased when angry
\end{enumerate}
\subsubsection{Culture}
\begin{enumerate}
\item Defined as the belief systems and values that influence customs,
  norms, practives and social institutions
\item Important to understand the influence of culture on an
  individuals' behavior
\item Multicultural Considerations:
  \begin{enumerate}
  \item Presentation of Symptoms
  \item Cultural meanings of illnesses
  \item Whether an illness is real or imagined, etc.
  \end{enumerate}
\item Gender Differences
  \begin{enumerate}
  \item Men and women may differ in emotional experience and
    expression such as insect phobia, alcoholism and eating disorders
  \item May be related to the social gender roles. 
  \end{enumerate}
\item Social Support
  \begin{enumerate}
  \item Low social support related to mortality, disease and psychopathology
  \item Especially important in the elderly
  \end{enumerate}
\item Social Stigma
  \begin{enumerate}
  \item May limit experession of mental health problems
  \item May discourage treatment seeking
  \end{enumerate}
\end{enumerate}
\subsubsection{Developmental Influences}
\begin{enumerate}
\item Life-Span developmental perspective
  \begin{enumerate}
  \item Addresses developmental changes
  \item Influence the presentation of psychological symptoms
  \item Influence and contrain what is normal and abnormal
  \end{enumerate}
\item The principle of equifinality
  \begin{enumerate}
  \item Form developmental psychopathology
  \item Several paths to a given outcome
  \item Paths vary by developmental stage
  \end{enumerate}
\end{enumerate}
\paragraph{Video 1}
What is a psychological disorder? We go by 3 Ds:
\begin{enumerate}
\item Psychological Dysfunction
  \begin{enumerate}
  \item A breakdown in cognitive emotional or behavioral functioning,
    anxiety vs panic attacks
  \item Limitation - These disorders exist on a continuum
  \end{enumerate}
\item Distress or impairment
  \begin{enumerate}
  \item Difficult in performing expected roles
  \item Limitations - Can be normal or expected and distress can be
    absent in a disorder, like anorexia
  \end{enumerate}
\item Deviates from average or cultural Norms
  \begin{enumerate}
  \item Limitation - Not all deviant behaviors are signs of mental
    disorder
  \item Some things can be atypical and not pathological
  \end{enumerate}
\item All of these by themselves do not determine illness, they all
  have to be considered at the same time when considering a disorder
\end{enumerate}
Here is the DSM-5 Definition:
\begin{enumerate}
\item A Mental disorder is a syndrone characterized by clinical
  significant disturbance in an individual's cognition, emotion
  regulation, or behavior thaat reflects a dysfunction in the
  psychological biological or developmental processes underlying
  mental function. 
\end{enumerate}
Psychopathology is the scientific study of psychological
disorders. Involves studying:
\begin{enumerate}
\item Clinical Description
\item Causation
\item Treatment and outcome -alleviate suffering
\end{enumerate}
Various mental health professionals study this way:
\begin{enumerate}
\item Ph.D
\item Psy.D
\item Ed.D : All of the above are mostly psychologists
\item M.D. : Psychiatrist
\item Psychiatric Social worker
\item Psychiatric Nurse
\item Family Therapist
\end{enumerate}
Adopt Scientist-Practitioner model by doing the following:
\begin{enumerate}
\item Consumers of science stay current with research in the field
\item Evaluators of Sciencd evaluate their own work and treatment
\item Creators of Science conduct their own research
\end{enumerate}
Historical Conceptions of Abnormal Behavior:
\begin{enumerate}
\item Supernatual
  \begin{enumerate}
  \item Agents outside outside our bodies cause these behaviors
  \item Evil spirits cause these, work of the devil , witches
  \item Barbaric treatments
  \item Exorcisms, torture and beatings among others 
  \item In the Renaissance, a Swiss psychologist attributed mental
    illness to gravity (inspired lunatic)
  \end{enumerate}
\item Biological
  \begin{enumerate}
  \item Mental illness is caused by circumstaance similar to psychial disease
  \item Hippocrates - Believed that all illness had natural causes,
    brain is seat of wisdom conscioussness intelligence and emotion.
  \item Galen - Linked abnormality with brain chemical imbalances, an
    excess or deficit in any of the body's fluids would cause mental illness
  \item Treatment involved regulating the environment, bloodletting
    and purging
  \item Renaissance - Determined that psychopathology has a
    medical/biological origin
  \item Rise of asylums which treat mentally ill
  \item Asylum treatments much of the same, involving restraints,
    purges and bloodletting.
  \end{enumerate}
\item Psychological
  \begin{enumerate}
  \item In early 19th century, Moral treatment came to be
  \item People with mental illness should be treated normally
  \item Pinel - Introduced these treatments along with Tuke and Rush
  \item Dorothea Dix started the Mental Hygiene movement
  \item Became dormant, but later Reemerged in three school of thought
  \end{enumerate}
\end{enumerate}
Here are the Three schools of thought of The Psychological Explanation
in the 20th century:
\begin{enumerate}
\item Psychoanalytic
  \begin{enumerate}
  \item Figures include: Sigmund Freud, Josef Breuer
  \item Behaviors stem from unconscious processes
  \item Treatment: psychoanalysis via hypnosis and dream analysis
  \end{enumerate}
\item Humanistic
  \begin{enumerate}
  \item Outgrowth of Freud
  \item Figures include: Carl Jung, Alfred Adler, Abraham Maslow, Carl Rogers
  \item Themes: Optimism about human nature, People should strive for
    self-actualization, people are innately good
  \item Treatment is Person-centered, conveys empathy, unconditional
    postive regard/accepatance
  \end{enumerate}
\item Behavioral
  \begin{enumerate}
  \item Figures include: Ivan Pavlov, Watson, BF Skinner
  \item Focuses on observable behavior and environmental determinants
  \item Scientific approach
  \item Less on what was going on in the mind and more what was
    happening in the environment to influence this
  \item Treatment is based in behavior as well
  \item Use behavioral principles to address disorders
  \item Focus on present
  \end{enumerate}
\item Modern views
  \begin{enumerate}
  \item Psychopathology is a product of multiple influences
  \item One-dimensional models are incomplete
  \item Defining abnormal behavior is complex
  \item Supernatural has no place in modern scientific world
  \end{enumerate}
\end{enumerate}
\end{document}
