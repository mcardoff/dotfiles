\documentclass[12pt]{article}
\usepackage{setspace}
\usepackage[margin=1.0in]{geometry}
\doublespacing

\title{\vspace{-5em}Shih, Chang, Chen Reaction Paper \\ The Model
  Minority Myth's Impact on Asian Americans}
\author{Michael Cardiff}
\date{July 21, 2020}

\begin{document}
\maketitle
The article in question is an excerpt from the Journal of Family Theory \& Review, Vol. 11, specifically Issue 3, which is on pages 412-428, it was published in September of 2019. This article mainly focuses on the family culture and "minority myth" stereotype of Asian Americans. The authors intend to show that there is much more nuance to the way which Asian Americans integrate themselves than at first looks. The author argues that Asian Americans are not one monolithic culture that experience the exact same path. The general stereotype is that Asian Americans tend to be very successful when they immigrate to the United States, they will be at the top in terms of education, salary, and other areas. The authors find this to be completely untrue, more specifically that the Asian American immigrant is much more diverse in their experience in the US.
\par
The authors are criticizing is the general view that Asian Americans
are in general more successful than other immigrant groups. This has
led to a broader general assumption that these immigrants do not face
discrimination. This comes from the fact that Asian Americans often
seem more likely to rise to the "same" social and economic status as
white people living in the US, so the discrimination which other
groups see are simply not present for Asian Americans. This is cited
as the elite minority myth. The authors counter this by citing the
sheer diversity in the Asian American population.
\par
The term Asian American refers to a population of over 30 subgroups,
which all together make up 6.4\% of the countries population(Shih, et
al., 412). The problem with this is that grouping all these people
together makes their experiences seem similar. This is obviously not
the case, the paper finds that there is as much diversity in Asian
American immigration than in any other group. This is particularly
problematic when it comes to research. Or at least it would be, if any
research was done. The author states that because of the myth of their
success and the general lumping together of all Asian Americans, there
is little research done in this group.
\par
This article is very strong when it comes to comparing its new ideas
to ideas of the past. Particularly the contrasting of William
Petersen's work with modern theory. The Petersen article praises
Japanese-Americans for overcoming the "adversity of their World War II
incarceration" in 20 years(Shih, et al., 414). This was shown to
directly conflict with a report by the Department of Labor in the
1960s, as both articles (Petersen and the DoL) cite the same reasons
for Asian American success and African American failure. The authors
connect this to the possibility that racism does not exist in
America. The authors state that the emphasis on the model minority
stereotype was simply constructed by the White Elite. This status as
simply an elite minority rather than a true American citizen prevents
members of this group from truly feeling like a member of the American
population, instead making them feel more a member of a group they had
left, more Asian than American.
\par
One idea that runs throughout the paper which is quite strong is the
idea of intersectional study. Since the group of Asian Americans
composes of so many groups, there are bound to be statistical errors
in looking at such a large group. The authors cite a myriad of
demographic data which shows the diversity in Asian ethnic groups. The
authors use disaggregated data (opposed to aggregated data) to show
the different trends for the various subgroups under the umbrella of
Asian American. The authors look at higher education, employment,
Household income, and Poverty. They find that many stereotypes of the
Asian Americans do not apply to all of these subgroups. Taking
education for example, when looking at the aggregate (all groups),
about 50\% percent of Asian Americans have a Bachelor degree. However,
looking at the individual groups, they find that over 70\% of
Taiwanese and Asian Indians have Bachelor's Degrees, while on the
lower end, less than 15\% of Laotians and Bhutanese have this higher
education (Shih et al., 416). The emphasis on a statistical argument
against the model minority myth aids the authors greatly.
\par
An interesting point the author makes in the beginning of the article
is that by 2055, Asian Americans are expected to become the largest
immigrant group in the US (Shih, et al.,412). This point is not used
much beyond its initial statement. I can understand why this would be
true, it is surprising that the authors never mention population
statistics much more. It seems that with their argument, it would make
sense to provide an updated stat. This would possibly show that there
would indeed be a large population of a few different Asian
Immigrants, but that one group would not dominate completely. This is
hardly a weakness and more of a nitpick, but it would still be nice to
see.
\par
This article was very interesting, and most definitely kept my
attention. Especially the discussions over the particular statistics
of groups when it came to gender. I found it interesting that Asian
American women made the most money out of any minority group, and even
then they only made 83 cents to the white man's dollar (Shih, et al.,
417). The authors' argument tended to be very modern. I found the
specific examples of Petersen and Moynihan to be helpful in
demonstrating why the posing of Asian Americans as an ideal is more
harmful than good.
\par
The only time the article really annoyed me is in the discussion about
Petersen. It almost romanticizes the hardships of the Japanese
Americans, it paints the struggles of these people as if it is some
fairy tale which got magically poof-ed away after World War II. As if
that was not bad enough, it was used to push a racist agenda that was
applied to all Asian Americans, which is simply nonsense. The fact
that the same arguments applied to African Americans to a completely
opposite effect was almost infuriating. What other concepts could be
explained in this way, or rather, what concepts HAVE been explained in
this way, to push a racist agenda.
\par
Overall, this article made me think a lot about the culture around
minorities in the US. Being a minority in the US is made out to be
some sort of label on your person which you could never get rid of,
and it forces people to work even harder to achieve the same status
that a white person almost automatically gets. Even if an immigrant
experiences these difficulties, which is expected, it is almost
guaranteed that this immigrant's child will experience those
difficulties as well. To add on top of this, the denial of this
struggle on the part of Asian Americans by presenting them as a model
minority, it almost begs for a change in the way minorities,
especially Asian minorities, are viewed by the US public. 

% model minority mean we racist
% People who do not have upward mobility its their fault, when it is
% systemic
% gender only analyzes women

\end{document}
