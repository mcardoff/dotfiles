\documentclass[12pt]{article}
\usepackage{setspace}
\usepackage[margin=1.0in]{geometry}
\doublespacing

\title{\vspace{-5em} Zingher Reaction Paper: An analysis of the
  changing social bases of America's political parties: Group support
  in the 2012 and 2016 presidential elections}
\author{Michael Cardiff}
\date{July 21, 2020}

\begin{document}
\maketitle
The article in question is taken from Elsevier's Journal of Electoral
Studies. It is not part of any specific volume of the Journal and
seems to originate online. The author of the paper is Joshua Zingher,
a professor of Political Science at Old Dominion University in
Virginia. The topic of this paper is electoral studies, given the
specific journal it is in. The author, like many people, found the
outcome of the 2016 election to be very odd, and the author thinks
that the cause of this may be from some sort of shift in the way that
people voted. The author had less of a thesis and more of an
investigative goal. He wanted to "assess how the party coalitions have
changed in the two most recent presidential elections" (Zingher, 1).
\par
The actual execution of the author's argument is a bit different than
how he initially set out. Despite saying that he will look at the two
most recent elections, the author begins with a discussion on how the
party demographic has changed over the past several decades
(Zingher, 2). This is an interesting choice by the author. On one
hand, it shows some uncertainty in his conclusion by not directly
paralleling his thesis, but on the other hand, it shows that his
argument is applicable outside the scope of the argument, which favors
the argument. Despite having an awkward start to the essay, the author
goes on to have a very mathematically driven discussion about the
party coalitions in these particular years.
\par
The bulk of the author's argument is mathematical in nature. There is
a statistical analysis, specifically a test of correlation, on a
specific group's likelihood to vote democratic given a certain group
membership (Zingher, 2). This type of analysis is very strong in
terms of being cross sectional, but can be subject to not being
complete. The author remedies this by determining politically relevant
groups, while this is not specifically defined, it is shown to be one
of either religion, income, race, ethnicity, education, age, gender,
and location. What I found to be inconsistent was some of the
multivariate combinations. The study found a need to distinguish
between White college graduates and White non-college
graduates. Perhaps if this was more the focus of the essay the author
would have put more care into this aspect of the cross-sectional
study. The author then goes into a discussion about more recent
elections and of the coalition changes there.
\par
The author talks mainly about the key changes in groups as to whether
or not they vote Democrat or Republican. He specifically notes that
there is not much change on either side. The most significant change
the author found was between 2012 and 2016, where White college
graduates went from voting democrat to republican (Zingher, 4). The
author also points out the relative stability in all other groups as
evidence against his argument of explaining the outcome of the 2016
election. It is at this point where the author turns to other
explanations. He suggests that instead of coalition factors, the
explanation might lie more in turnout, as the turnout for African
American voters in 2016 was down from 2008 and 2012 (Zingher,
4). While this article is definitely mathematically sound, this is
less true when it comes to other biases. 
\par
One thing I found persistent through this article is a slight bias to
the democratic side. This is present in representing the democratic
regression as positive as opposed to negative (Zingher, 2). There is
no way to avoid a bias this way, especially if there are only two
options and one is (literally) negative. Despite this, it still does
question the view of only the white college graduates being a recent
turnover to democrats. Whether or not this is because of a bias is up
for debate. The numbers for WCG in Table 2 on page 4 are very similar
between 2000 and 2016, so this calls into question what the article is
saying about the democratic party. Using similar logic, we can say
that between 2008 and 2016, the protestants have been leaning more
towards the democratic side, even if it is very much still heavily on
the republican side and the difference was only by 5\%. Despite the
very few flaws, I definitely enjoyed this article. 
\par
This article was definitely very interesting for me. Coming from a
more mathematically intensive major, it was refreshing to see some
math in an article, even if it was limited. The concept of a
regression and correlation coefficients was more familiar to me than
a discussion about the constitution or public opinion. This made for a
fairly engaging read. This allowed for me to better understand the
argument and how it evolved throughout the article. This article also
changed how I personally understood the 2016 election. 
\par
This article definitely better clarified the 2016 elections. Before I
read this article, the election seemed to almost be a mystery. The
talks of possible interference mystified it even more. However, by
reading this article, it seems more and more that the election was
more normal than anything else, and that previous elections were more
the outliers. The explanation of declined African American turnout
made so much more sense than anything I heard on the news about this
election. Overall, this definitely had to be my favorite article from
the semester so far, as it feels as if the clouds have been lifted on
so many odd concepts. 
\end{document}
