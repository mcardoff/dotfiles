\documentclass[12pt]{article}
\usepackage[margin=1.0in]{geometry}

\title{Exam 1}
\author{Michael Cardiff}
\date{July 9. 2020}

\begin{document}
\maketitle
\section*{Question 1}
\subsection*{Question}
What is the difference between a representative democracy and a direct democracy. and
where do we see examples of each in the American political system?
\subsection*{Answer}
\indent A direct democracy has the citizens participate directly in the government. In a
representative Democracy, the people do not directly influence the government. The people
instead will elect representatives (hence the name) that may be more familiar with the
semantics of the government (Krutz et. al, 15). The idea is that these people will be more
efficient in choosing what happens to the people of the government. There are examples of
both types of democracy in the US.
\paragraph{}
\indent The much more common form of democracy seen in the US is the representative
democracy. At the federal level, both the senators (in the Senate) and representatives (in
the House) are elected by the people. The people also partake in the election of the
President, however this process involves a complex system known as the electoral
college. In Illinois at the state level, there is a similar structure. The Governor is
elected by popular vote, and instead of a house of representatives there is a general
assembly. They accomplish much of the similar duties at the state level as opposed to the
federal level. Even at the city level in Chicago, there is an elected mayor and aldermen
and alderwomen which make up a city council. All of these are examples of governmental
bodies run by representatives elected by the people (Krutz et. al, 16).
\paragraph{}
\indent Examples of direct pariticipation in democracy are few and far between in the
US. One that immediately Comes to mind is the New England town hall meeting (Krutz et. al,
16). These meetings allow people to discuss their problems and speak their stresses about
what is happening in the town. These aspects exist outside of just New England. People are
often encouraged to write to their senators to encourage support for a certain bill. This
ensures that the senators have a connection with the people that they are supposed to
represent. 
\newpage
\section*{Question 2}
\subsection*{Question}
What shortcomings of the Articles of Confederation was the Constitution of 1787 designed
to fix, and did it?
\subsection*{Answer}
The Articles of Confederation had many shortcomings which provided the early US with a
weak central government which could not do much. Early foreign relations were a mess, as
Congress had no power to impose tax on the people. This was because they feared taxing the
people without fair representation (Krutz et. al 45). While this resulted in a small
central government, it was way too small to be effective to any extent. One final problem
was the supremacy of the state over the central government. This meant that when the
federal government asked the states for money, they could deny that money. This lead to an
underfunded central government, which cannot get anything done. 
\paragraph{}
\indent The constitution fixed some problems simply by giving the government more
powers. For example the problem of an underfunded government was solved by the ability by
Congress the enumerated power to levy tax on the people. In order to prevent the fear of
overstepping, the drafters of the constitution included the general welfare clause to
indicate that no law should be passed which hurts the American people. The constitution
was about balance, specifically with the state and central governments. This led to the
supremacy clause of the constitution. This clause stated that the constitution would be
the 'supreme' document of US law (Krutz et. al, 54). If any state law were to contradict a
concept framed by the constitution, then the constitution's ruling should prevail. Even
then, the framer's believed this would create too big of a government. This makes sense, as
the constitution could literally trump any single state law. The solution here is the
creation of separate powers in the constitution which balance each other so neither end is
too powerful and cannot infringe upon the rights of the states.
\newpage
\section*{Question 3}
\subsection*{Question}
How has political culture been used to explain the evolution of American politics? Why is
it impossible to use public opinion to explain that evolution?
\subsection*{Answer}
The political culture is a set of rigid, shared views and judgements held by a group of
people with regard to the system of politics. Public opinion on the other hand is not
specific to the system of politics, it involves the people who participate in
politics. People expressing their support for one president over another are expressing
their public opinion, whereas people who demonstrate support (or lack thereof) for a
certain bill may be expressing the current political culture. The public opinion is
inherently fluid, so it changes rapidly and in many different ways. The political culture
is much more rigid, but can change over longer periods of time.
\paragraph{}
\indent The change in political culture can correspond with a change in ideals. If an
event occurs which completely contradicts the views of people at the time, a shift is
necessary in political view. The new political culture will reflect the needs of people in
that time. One example is the shift in the structure of congress following the stock
market crash of the late 1920s (Krutz et. al, 203). Before the crash, many citizens had
voted in Republican representatives and senators, this changed quickly however as the
people felt the democratic party's policies had suited their needs better at the
time. What did not matter was the people's fluid opinion on who the specific people were
that were running the show. All that mattered to the people were what exactly was being
done to ensure their suffering was minimized. The people did care in that time who was
doing it, but overall it was more the policies that enticed the people rather than the
specific people.

\end{document}
