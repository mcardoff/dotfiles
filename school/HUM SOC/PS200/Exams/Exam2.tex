\documentclass[12pt]{article}
\usepackage{setspace}
\usepackage[margin=1.0in]{geometry}
\doublespacing

\title{\vspace{-5em} Exam 2}
\author{Michael Cardiff}
\date{July 23, 2020}

\begin{document}
\maketitle
\section*{Question 1}
\paragraph{Q}
Why did the Framers set up such a complex and difficult process for
legislation to become law, and what have been the positive and
negative consequences?
\paragraph{A}
The Framers of the Constitution did not want it to be easy for things
to change in the US, so they set up a very complex process for
legislation to become law. They wanted to prevent any drastic changes
from being put through to law too quickly. Coming from a monarchy as
well, they wanted to avoid a tyrannical government such as the one
they came from. This process, while ensuring checks and balances on
each branch of government, has shown its especially in recent,
regarding the current actions of Senate Majority Leader Mitch
McConnell.
\par
The Framers intentions when making such a complex system of turning
legislation into law was for a few reasons. Mainly, they wanted to
ensure that the status quo was maintained and that the government was
kept in check and so that no tyrannical government would rise. It is
very effective in doing both of these, due to the many roadblocks
which the legislature faces on it way to becoming a law. The classic
legislative process has the bill starting in either house of Congress
but it must be approved by both before it even gets a chance to be a
law. Even after that, the Executive Branch in the form of the
President, must check on the legislature and sign on it. After this,
it still must be deemed constitutional by the courts (Krutz et al.,
436). This does not even mention much of modern legislation, which
involves committees and much more discussion of the intricacies of the
bill such as its cost, amending, and much more. These ensure checks
and balances as at every stage of the bill's 'life' there are points
where it can be removed by another branch of the government. Since
this process has so many steps, there is a guarantee that no spur of
the moment decisions will be made based on small changes in the
current environment. By the time it came for the other House of
Congress to vote on the legislation, those conditions will have worn
out, and it will no longer be necessary. That is even if it makes it
out of the first House of Congress in the first place. While this is a
big strength of our legislature system, there are some steps which
lead to a rather stagnant government.
\par
One main weakness that can be seen in the legislative process is to
ignore it. Senate Majority Leader Mitch McConnell has a lot of bills
which were passed by the House of Representatives and he has yet to
present to the Senate. It is his job to present these bills to the
senate body, yet he has not. On one hand, he is ensuring that these
bills were not too hastily passed by the House, but on the other hand,
some of these bills have been on his desk for quite a long time, so
there has been plenty of time for the conditions to die down, so why
not place these bills on the floor to get voted down? Since the House
is currently majority Democrat, it points to a partisan issue rather
than an issue of passing legislation. This is not good, essentially
placing a brick wall on all legislation coming from House Democrats,
so truly nothing can get done on their part. The weakness that emerges
is less with the actual implementation by the Framers, but rather with
its practice especially in the emergence of a partisan system.
However, this must be taken further, is this not just another form of
filibuster? The problem with seeing these actions as a form of
filibuster is that it is not the lesser represented party preventing
the larger from passing their own legislation, it is the other way
around, which is dangerous for a system which heavily relies on the
smaller power being able to pass its own legislation. 
\section*{Question 2}
\paragraph{Q}
What are the opportunities and limitation for presidential leadership
in the contemporary US political system?
\paragraph{A}
The Framers of the Constitution initially gave the President very few
powers, afraid that they would be too much like a king, which they
desperately wanted to avoid (Krutz et al., 447). Over the years
however, the president has had many more opportunities to increase its
leadership. The increase in the size and influence of the position of
the president comes from various international problems.  These
problems have lead to the office of the president to become more
involved in all aspects of politics, national and international.
\par
The president is seen as in to the World as the figurehead of United
States politics. This means that much of international relations is
headed by the President. This can be traced all the way back to when
George Washington's Cabinet issued a statement of neutrality in 1793,
proclaiming that the US would be neutral in the current European
conflict between France and England. The president did this despite
any declarations of war needing to originate from the House. Many
other war-times are when we see President's extend their power in this
way. The position of the president as the Commander-in-Chief of the US
Army has led to much of this misuse of power.
\par
There still are limitations to the President's powers, most of which
were been defined in the Constitution. More specifically, Congress has
the power to impeach the President if they find him/her unfit to the
position. This was seen most recently when Donald Trump was impeached
by the House of Representatives. Before this, both Bill Clinton and
Andrew Johnson have been impeached (Krutz et al., 227). However,
none of these cases have led to the President being removed from
office, so this seems to be quite a weak limitation. 
\section*{Question 3}
\paragraph{Q}
In what ways is the U.S. court system better suited to protect the individual than
are the other branches of the government?
\paragraph{A}
The courts more often than not, will deal directly with
individuals. When you hear about landmark achievements by the supreme
court, you will hear the names of the people (i.e. Marbury and
Madison) rather than the name of a bill (i.e. Civil Rights Act of
1964)(Krutz et al., 488). The conditions on which Supreme Court
Justices are chosen are very different than elected officials, and the
course of their careers vary as well. These reasons together provide a
much different dynamic when looking at the law of the US.
\par
The court system will quite literally deal with the people. This
allows for the courts to hear the issues that people have directly as
opposed to hearing in the form of public outcry that elected officials
might have to do. With the structure of the courts which we have, if a
case is able to reach the Supreme Court, then there is some major form
of problem which requires an in-depth look at the laws presented by
the Constitution.
\par
The life-long term of a Supreme Court Justice completely removes the
worry of re-election that some elected officials will have. This
encourages the Justices to take on more definitive stances that will
result in the betterment of the condition of the people, rather than
having to appeal to a party, or even the people themselves. Take the
decision from Brown v. Board (1954), despite segregation existing in
many of the Southern States, the courts had deemed that separate but
equal was inherently unequal (Krutz et al., 1954). This was a
controversial opinion, but it was made to the benefit of the people,
rather than some party or other elected official. The detachment from
a fear of reelection allows the Justices in the Supreme Court to be
more impartial in their decisions in the laws of the US.
\section*{Question 4}
\paragraph{Q}
Why does it make sense to refer to the bureaucracy as the “fourth branch” of the
U.S. government? Why does it not make sense?
\paragraph{A}
The bureaucracy is an important gear in the structure of the American
government, and without it, the government would not be able to
function in the way we know today. However, the extent to which it can
be considered a fourth branch of government is debatable. On one hand,
it makes sense in a literal way. The bureaucracy is an extension of
the federal government which take charge of certain tasks given by the
federal government. On the other hand, there are certain implications
of another 'branch' of government which is not truly enforceable.
\par
The bureaucracy is like a fourth branch of government in that it is an
extension of the federal government. The bureaucrats in charge of it
are an essential part of the connection between the federal government
and the people. The bureaucracy provides a way for the people to
insert themselves into the government, while still limiting the
content which actually gets passed on to the federal government (Krutz
et al., 558). In these ways it is in fact a fourth branch of the
government.
\par
While in a literal sense the bureaucracy is a fourth branch of
government, the implications of a 'branch' is where this argument
fails. Specifically when it comes to checks and balances this argument
becomes difficult. In a way, the bureaucracy is a check or balance for
the people, as mentioned before, but there is very little in terms of
proper checks and balances. This is because the bureaucracy is a
byproduct of other federal branches, rather than one which is
independent from it (Krutz et al., 559). The judiciary branch for
example, has connections to the federal government in that its members
must be appointed by the Executive and approved by the Legislative,
but its action is completely independent of actions done by the other
branches. This is in contrast to the bureaucracy, in which entire
positions could be removed without contest by the federal government,
and the public would not necessarily be effected too much.
\section*{Question 5}
\paragraph{Q}
Which civil liberties has the federal government done a good job of protecting,
and which ones could it better protect?
\paragraph{A}
The Government, if viewed in terms of the constitution, is very good
at protecting the rights of persons who have not yet been tried. Many
of the amendments in the bill of rights have to do with rights
pertaining to a trial (Krutz et al., 117). Even past this, there have
been more rights established such as the Miranda rights, which make it
clear to the people that they are protected by the United
States. Along with this, the rights of people in regards to protecting
themselves. The laws related to guns in the US are meant to help
people in protect themselves. The US government is very good at
ensuring the protection of its people from unfair imprisonment, as
well as protection from various attacks on their person.
\par
Despite this, the civil liberties and rights that have to do with race
and sex have been very poorly protected. There are innumerable court
cases which have to do with both of these, and an equal amount of
legislation passed. Despite this, there is still more which people
believe needs to be passed. Even with the amendments to the
Constitution as well, there still is discrimination against other
people due to their race and sex. Cases like Brown v. Board have
happened, yet soon after there was a presidential candidate who won
many votes in Southern States solely on the platform of
segregation. Even with the unconstitutional ruling of discrimination
on the basis of sex, women still make less money than men across the
board. These liberties and rights are much more important to protect
with the people, and it is a major failure that these still exist at
the state level. 
\end{document}
