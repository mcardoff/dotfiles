\documentclass[12pt]{article}
\usepackage{mathptmx}
\usepackage{setspace}
\usepackage[margin=1.0in]{geometry}
\doublespacing

\title{\vspace{-5em} Exam 3}
\author{Michael Cardiff}
\date{August 6, 2020}

\begin{document}
\maketitle
\section*{Question 1}
\paragraph{Q}
What explains voter turnout in the U.S. (overall and across groups and
regions), and why do many observers consider it a cause for concern?
\paragraph{A}
Many things affect voter turnout in the US, but a lot of them depend
on specific groups of people, but there are a few that span
overall. When it comes down to it, you cannot show up to the polls if
you are prevented from either registering or actually voting when the
day comes. In the same vein, if you are not required to show up, many
people simply will not show up. In terms of specific groups, a more
specific problem lies in discrimination against certain minority
groups and people of certain socioeconomic status.
\paragraph{}
Many people in the US simply do not see the incentive to vote. Voting
participation is not mandated in the least, so it slips their mind
when it comes to election day. Places where voter turnout is high such
as Turkey and Belgium place fines on people who do not vote (Krutz et
al., 254). The incentive in this case for voting is that you will not
have to pay this fine, hence people vote more often. This can however
be seen as a cause for concern, as the voters might seem more
threatened to vote rather than voting out of their own free will.
\paragraph{}
Some people are prevented from showing up to the polls at
all. Convicted felons are completely barred from voting, but this does
not explain low voter turnout as these felons would not be part of the
VEP. The type of prevention here have to do more with the process of
voter registration. Requiring a photo ID can be a major restriction as
it unfairly targets certain groups (Krutz et al., 255). Most of the
time you have to get a photo ID from the DMV, and the first thing
everyone thinks of when you mention DMV is long lines. This means
people who do not have a lot of money and need to work two or more
jobs will not have time to go get an ID. Even then, if people have
time to get an ID, in some places they cost money, though it is often
not a lot, it is still money that could be put to better use.
\section*{Question 2}
\paragraph{Q}
What is the institutional argument and the cultural argument for why third
parties do not perform well in the U.S.? 
\paragraph{A}
Third parties in general do not gather much support in the US
political system. Neville-Shepard mentions that other political
scientists often attribute this to Duverger's law (Neville-Shepard,
275). This law is a statement about the nature of the plurality voting
system. Neville-Shepard counters this point by saying it is more of a
problem with how the general public views these third party
candidates. These candidates are less viewed as alternatives and more
as invaders in the election of two parties.
\paragraph{}
Duverger's Law is the more institutional view of third parties. It
specifically states in a plurality rule system with single-member
districts (like the US), there tends to be a two-party system. This
argument relies on the "mechanical" effect that voting for a third
party has on the "system" (Neville-Shepard, 275). By mechanical
effect, this is a sort of implication that a third party would ruin
how the system is currently working, and would result in a system that
would get nothing done. This in turn, discourages the voters from
voting third party.
\paragraph{}
The argument which Neville-Shepard presents is that third party
candidates are presented as "invaders" to the election
(Neville-Shepard, 280). Many people during the course of the 2016
election mentioned that a vote for a third party is as good as staying
out of the election (Neville-Shepard, 281). Even today, people who are
vehemently against one candidate or the other in the current election
will say that a vote for a third party is a vote for insert
unfavorable candidate here. So the language surrounding third parties
is less regarding a serious alternative to the current system, but
instead a thrown away vote that contributes to the person you do not
want to win instead of the person you voted for. 
\section*{Question 3}
\paragraph{Q}
Which electoral reforms have been proposed to resolve the U.S.’ political
dysfunction, and which changes seem to be at least possible? 
\paragraph{A}
There have been many reforms suggested to change how the current
electoral system functions. One of the more common suggestions is to
abolish the electoral college of the president. An alternative to this
is suggested in the article by Foley, mentioning that we could return
to the Jeffersonian method of giving out electoral college votes. In
terms of general non-presidential elections, there are alternative
voting systems which could be used to combat the current deadlock. One
suggestion is a ranked voting system, in which the voters specifically
rank their choices for the person they are voting for.
\paragraph{}
In terms of other general voting systems, ranked choice voting seems
like the most sensible alternative, but it is not likely to be
implemented. A ranked choice system would put less stigma on third
party candidates and would solve the previous problem of a throw away
vote when it comes to third parties. But the real problem occurs when
no majority is obtained by any candidate. This does give any other
candidate a chance to win, but the instant runoff in the form of
counting the ranked choice will give many votes to candidates who did
not have many in the first place. This system seems very confusing and
I could see people going into even more outrage because the candidate
they voted for did not win (even if that happens now). 
\paragraph{}
In terms of presidential elections, the current system we have allows
for presidents to be elected without being the majority choice of the
people. In fact, Presidents Trump, Bush, and Clinton all did not win
the popular vote (Foley, par. 23). Foley suggests that we should
instead return to the original Jeffersonian style of assigning
delegates. This process would ensure the person is described as a
commitment to majority rule throughout the article (Foley,
par. 4). This differs from a direct popular vote as it still relies on
the districts to individually vote, but it also relies on the overall
majority being in one person. A direct popular vote would simply place
too much power in the hands of the people, which is against what the
Framer's of the constitution wanted. This system seems to have the
biggest chance of being actually implemented.
\section*{Question 4}
\paragraph{Q}
What research methods have been used in the articles assigned for Sessions 8,
9, and 10, and what are some of their strengths and weaknesses?
\paragraph{A}
The three articles used for Sessions 8 9 and 10 use very different
methods to study similar phenomena. Session 8's reading by Pied mostly
uses survey and polling to gather data. The Zingher reading uses
statistical analysis of already gathered data. The Neville-Shepard
paper uses a specific case study as its method of research. All of
these papers talk in some way about the 2016 Presidential Election, so
it is interesting to compare the effectiveness of these methods.
\paragraph{}
%Pied
Pied uses existing polls along with an ethnographic research method in
order to demonstrate his argument. The elements of ethnography are
very strong, as it allows for the interviewing of specific people
involved in the events which the author is concerning himself with
(Pied, 774). The main issue with this type of research is the presence
of the actual researcher. Often in ethnographic studies the researcher
tries to make themselves relatively unknown, but when it comes to
interviews, this is not always the case. The presence of the
interviewer can influence what people say. What makes up for this on
the part of the author is the polls, which are something more concrete
that are less subject to this bias. 
\paragraph{}
% Zingher
Zingher's analysis of the various party demographics is strictly
mathematical in nature. He uses a statistical regression test on
the likeliness that someone would vote democrat in various election
years (Zingher, 2). Later on in the article he goes on and uses a
group contribution percentages to see what percent of the
democratic/republican vote a particular group made up. This method is
very effective as it provides a rule of analysis which is very
generalizable, so many of the groups which are found in the US can be
properly represented. A weakness of this method however is that it is
hard to tell anything about a single election year from a data set, it
is more reliant on comparing to previous data as seen in the table on
Page 4. 
\paragraph{}
%Neville-Shepard
Neville-Shepard uses employs an interesting analysis of published
media in the time near the 2016 election. The analysis seems almost
linguistic in the way that he describes the way that the media talks
about the third parties (Neville-Shepard, 281-282). This method,
unlike Zingher's, is much more effective in its use of the general
pattern it finds. Instead of letting the research find the pattern,
the pattern is established early in the paper, even in the abstract,
and then expanded upon using evidence from not only the most recent
election year, but also previous controversial years such as the 2000
election. However, media in general has a very large party bias,
almost all news stations or papers have a corresponding party alliance
which most people know about. 
\section*{Question 5}
\paragraph{Q}
How have domestic policy-making and foreign policy-making differed in the
U.S., and how have they been connected? 
\paragraph{A}
Even though domestic and foreign policy share some similarities, they
differ for key reasons that make the US very unique. One way that they
are different is simply in the content which they deal with. Domestic
policy more often deals with the wants and needs of the people where
foreign policy is dealing with the wants and needs of other
nations. One connection between them is the people who carry out these
actions, and this is what makes the connection unique to the US.
\paragraph{}
The differences are quite stark when it comes to domestic and foreign
policy. Foreign policy often will involve a larger amount of political
figureheads over a longer amount of time (Krutz et al., 628). Domestic
policy on the other hand will involve strictly the head of state
governments and their interactions with the federal government over
shorter periods of time (Lynch,Gramer, par. 1). Domestic policies may
have to do with how individual states are handling the COVID-19
pandemic. This involves whether or not the federal government is
requiring people to wear masks, as well as the corresponding
state-level orders as well.
\paragraph{}
The similarities come when it comes down to who is affected. The
president, as both head of state and head of the nation has a large
role in both domestic and foreign policy (Lecture 7/28). In any
article you will see regarding US politics, there is bound to be at
least one mention of either the US president, congress, or the supreme
court. All of these different branches of the government will
intertwine with one another and will effect one another, as they are
serving not only the domestic needs, but also its international
needs. 
\end{document}
